\documentclass[12pt,a4paper]{article}

% Márgenes
\usepackage[
  left=3cm, right=3cm,
  top=3.2cm, bottom=3.2cm
]{geometry}

% Idioma y codificación
\usepackage[spanish, es-nodecimaldot]{babel}
\usepackage[utf8]{inputenc}
\usepackage[T1]{fontenc}

% Utilidades
\usepackage{graphicx}
\usepackage{ragged2e}
\usepackage{setspace}
\usepackage{microtype}
\usepackage{amsmath,amssymb}
\usepackage{float}
\usepackage{booktabs}
\usepackage{siunitx}
\usepackage{caption}
\usepackage{subcaption}
\usepackage{hyperref}
\usepackage{times}
\usepackage{listings}

% Configuration
\sisetup{
    separate-uncertainty=true,
    output-decimal-marker={.}
}

\hypersetup{
    colorlinks=true,
    linkcolor=black,
    citecolor=black,
    urlcolor=blue
}

% Commands for vectors (bold, non-italic) and scalars (italic)
\renewcommand{\vec}[1]{\mathbf{#1}}

% Regla fina para separador
\newcommand{\ThinRule}{\noindent\rule{\textwidth}{0.4pt}}

% Regla de notas al pie corta
\makeatletter
\renewcommand{\footnoterule}{%
  \kern-3pt
  \noindent\rule{5.5cm}{0.4pt}\par\kern 6pt
}
\makeatother

% Ajustes de párrafo
\setlength{\parindent}{0pt}
\setlength{\parskip}{0.6em}
\renewcommand{\contentsname}{Índice}

% Configuración de listings
\lstset{
  basicstyle=\ttfamily\small,
  keywordstyle=\bfseries,
  breaklines=true,
  frame=single,
  columns=fullflexible
}

% Helper para incluir figuras solo si existen
\newcommand{\includegraphicsifexists}[2][]{%
  \IfFileExists{#2}{%
    \includegraphics[#1]{#2}%
  }{%
    \fbox{\parbox{0.9\linewidth}{\centering \textbf{Falta generar resultado.}\\
    Genere el archivo \texttt{#2} ejecutando los scripts correspondientes.}}%
  }%
}

\begin{document}

\begin{titlepage}
\centering

% --- Logo ITBA centrado ---
\vspace*{0.3cm}
\includegraphics[width=4.2cm]{assets/logo-itba.png}\par

% --- Regla separadora ---
\vspace{0.6cm}
\ThinRule

% --- Título ---
\vspace{0.9cm}
\begingroup
\fontsize{18}{19.2}\selectfont
\textbf{Análisis GTFS Vancouver}\par
\endgroup

\vspace{0.4cm}
{\large Trabajo Final}\par
\vspace{0.25cm}
{\large 73.82 - Bases de Datos Espaciales y de Movilidad}\par

% --- Autores ---
\vspace{1.1cm}
{\normalsize \textbf{Autores:}}\par
\vspace{0.25cm}
{\large \textbf{Tomas Camilo Gay Bare\textsuperscript{1}}}\par
\vspace{0.15cm}
{\large \textbf{Manuel E. Dithurbide\textsuperscript{2}}}\par

% --- Fecha ---
\vspace{1.4cm}
{\normalsize Diciembre 2025}\par

% --- Notas al pie ---
\vfill
\footnotetext[1]{tgaybare@itba.edu.ar}
\footnotetext[2]{mdithurbide@itba.edu.ar}

\end{titlepage}

\tableofcontents
\newpage

\section{Introducción}

Este proyecto analiza el sistema de transporte de Vancouver utilizando datos GTFS estáticos
y GTFS-Realtime de TransLink. Se implementa un pipeline reproducible sobre PostgreSQL con
extensiones PostGIS y MobilityDB, utilizando Python para análisis y visualización.
El pipeline permite caracterizar la red estática, comparar planificación vs.\ operación
y generar visualizaciones para identificar problemas de servicio.

\section{Metodología de replicación}

En esta sección se documenta paso a paso el pipeline completo, desde la instalación
hasta la generación de los resultados presentados en las secciones siguientes.

Nota: todos los datos utilizados para este análisis pueden encontrarse en 
\url{https://drive.google.com/drive/folders/19Hic_CPM1u5DPWin90Dc3BLOUnqnwCB3?usp=sharing}.

\subsection{Instalación de dependencias}

Situarse en la raíz del repositorio y ejecutar:

\begin{lstlisting}[language=bash]
pip install -r requirements.txt
pip install -r static_analysis/requirements.txt
pip install -r realtime_analysis/requirements.txt
npm install -g gtfs-via-postgres
\end{lstlisting}

Configurar la conexión a la base de datos exportando las variables de entorno:

\begin{lstlisting}[language=bash]
export PGHOST=localhost
export PGPORT=5432
export PGUSER=postgres
export PGPASSWORD=postgres
export PGDATABASE=gtfs
\end{lstlisting}

Para hacer estas variables persistentes, agregarlas al perfil del shell
(e.g., \texttt{\textasciitilde/.bashrc}, \texttt{\textasciitilde/.zshrc}).

\subsection{Requisitos de la base de datos}

Antes de ejecutar cualquier comando, asegurarse de que:
\begin{itemize}
  \item La base de datos \texttt{gtfs} existe y está corriendo.
  \item Las extensiones \texttt{PostGIS} y \texttt{MobilityDB} están instaladas y habilitadas.
  \item Las variables de entorno de conexión están exportadas (ver sección anterior).
\end{itemize}

Para un listado completo de prerrequisitos (incluyendo instalación de GDAL/rasterio/contextily),
revisar la sección \textbf{Prerequisites} del \href{../README.md}{README.md} del proyecto.

Se puede verificar la conexión y las extensiones con:

\begin{lstlisting}[language=bash]
psql -h $PGHOST -p $PGPORT -U $PGUSER -d $PGDATABASE \
  -c "SELECT PostGIS_version(), MobilityDB_version();"
\end{lstlisting}

\subsection{Flujo GTFS estático}

\subsubsection{Descarga y preprocesamiento}

Ejecutar el script de descarga y limpieza:

\begin{lstlisting}[language=bash]
bash static_analysis/data/download_data.sh
\end{lstlisting}

Este paso genera los archivos prunados en \texttt{static\_analysis/data/gtfs\_pruned/}.

\subsubsection{Importación de tablas GTFS}

Con la base de datos en marcha, importar las tablas:

\begin{lstlisting}[language=bash]
cd static_analysis/data/gtfs_pruned
gtfs-to-sql --require-dependencies -- *.txt \
  | psql -h $PGHOST -p $PGPORT -U $PGUSER -d $PGDATABASE
\end{lstlisting}

\subsubsection{Creación de estructuras MobilityDB}

Desde la raíz del repositorio:

\begin{lstlisting}[language=bash]
cat static_analysis/data_loading/mobilitydb_import.sql \
  | psql -h $PGHOST -p $PGPORT -U $PGUSER -d $PGDATABASE
\end{lstlisting}

Este script genera tablas como \texttt{scheduled\_trips\_mdb},
\texttt{route\_segments}, vistas agregadas y columnas geométricas.

\subsubsection{Carga de densidad poblacional}

Para los análisis que combinan densidad poblacional y cobertura de transporte,
se utiliza un GeoJSON de áreas censales de Vancouver. El siguiente script
calcula la densidad (\texttt{pop}/\texttt{a}) e importa la capa
\texttt{population\_areas} en la base de datos:

\begin{lstlisting}[language=bash]
python static_analysis/data/download_population_data.py \
  --geo static_analysis/data/population/vancouver_geo.geojson
\end{lstlisting}

\subsubsection{Consultas y visualizaciones estáticas}

Primero, generar las vistas materializadas necesarias para los análisis:

\begin{lstlisting}[language=bash]
python static_analysis/queries/sql/run_sql.py
\end{lstlisting}

Luego, ejecutar el orquestador que refresca las vistas y genera las gráficas:

\begin{lstlisting}[language=bash]
python static_analysis/queries/run_all_analyses.py
\end{lstlisting}

Este comando:
\begin{itemize}
  \item Ejecuta los archivos SQL en \texttt{static\_analysis/queries/sql/},
        creando vistas materializadas para visualización en herramientas de mapeo
        (densidad de rutas, segmentos de velocidad, población vs.\ transporte, etc.).
  \item Ejecuta los scripts de \texttt{static\_analysis/queries/visualizations/},
        generando gráficos en formato PNG (histogramas, distribuciones por ruta,
        comparaciones de velocidad, métricas de proximidad a estadios, etc.).
\end{itemize}

Los resultados se guardan en \texttt{static\_analysis/queries/results/},
organizados por tipo de análisis.

\subsection{Flujo GTFS-Realtime}

\subsubsection{Configuración y tablas realtime}

Instalar dependencias adicionales y configurar la API de TransLink:

\begin{lstlisting}[language=bash]
export TRANSLINK_GTFSR_API_KEY="your-translink-api-key"
\end{lstlisting}

Crear las tablas necesarias para datos realtime:

\begin{lstlisting}[language=bash]
cat realtime_analysis/data_loading/realtime_schema.sql \
  | psql -h $PGHOST -p $PGPORT -U $PGUSER -d $PGDATABASE
\end{lstlisting}

\subsubsection{Ingesta de feeds GTFS-Realtime}

Ejecutar el proceso de ingesta durante un intervalo de tiempo desde el root del repositorio:

\begin{lstlisting}[language=bash]
python -m realtime_analysis.data.ingest_realtime \
  --duration-minutes 20 \
  --poll-interval 30
\end{lstlisting}

Este comando consulta periódicamente los endpoints de posiciones y actualizaciones de viaje,
almacenando los mensajes en tablas \texttt{realtime\_*}.

\subsubsection{Construcción de trayectorias reales}

A partir de los puntos GPS se construyen trayectorias \emph{map-matched} sobre los
recorridos estáticos:

\begin{lstlisting}[language=bash]
python -m realtime_analysis.data.build_realtime_trajectories
\end{lstlisting}

Las trayectorias resultantes se almacenan en \texttt{realtime\_trips\_mdb}.
Los puntos GPS crudos se deduplican y se ajustan (\emph{snap}) a las formas
programadas antes de insertarse en la tabla.

\subsubsection{Consultas y análisis agregados con GTFS-Realtime}

Primero, generar las vistas materializadas para los análisis:

\begin{lstlisting}[language=bash]
python realtime_analysis/queries/sql/run_sql.py
\end{lstlisting}

Luego, ejecutar el orquestador que refresca las vistas y genera gráficos:

\begin{lstlisting}[language=bash]
python -m realtime_analysis.queries.run_all_analyses
\end{lstlisting}

El comando anterior ejecuta los archivos SQL de \texttt{realtime\_analysis/queries/sql/},
creando vistas \texttt{realtime\_*} y vistas de visualización, y luego corre
los scripts para producir gráficos en PNG y resúmenes en CSV.

También es posible ejecutar cada análisis de forma individual:

\begin{lstlisting}[language=bash]
cd realtime_analysis/queries/visualizations
python speed_vs_schedule_analysis.py
python schedule_times_analysis.py
python delay_segments_analysis.py
python headway_analysis.py
\end{lstlisting}

Los resultados se almacenan en \texttt{realtime\_analysis/queries/results/},
organizados por tipo de análisis.

\section{Análisis de GTFS estático}

\subsection{Composición y alcance de la red}

La primera consulta caracteriza la composición de la red en términos de tipo de servicio
(\texttt{bus}, \texttt{subway}, \texttt{rail}, \texttt{ferry}, etc.), utilizando las
tablas GTFS \texttt{routes} y un CTE de mapeo de códigos para calcular el porcentaje
de rutas por modo.

\begin{lstlisting}[language=SQL]
WITH route_types(route_type, name) AS (
  SELECT '0', 'streetcar' UNION
  SELECT '1', 'subway'    UNION
  SELECT '2', 'rail'      UNION
  SELECT '3', 'bus'       UNION
  SELECT '4', 'ferry'     UNION
  SELECT '11', 'trolley'
),
route_groups AS (
  SELECT 
    route_type,
    COUNT(*) AS qty,
    ROUND(COUNT(*) * 100.0 / SUM(COUNT(*)) OVER (), 2) AS perc
  FROM routes
  GROUP BY route_type
)
SELECT name, qty, perc 
FROM route_groups g JOIN route_types t ON g.route_type = t.route_type
ORDER BY perc DESC;
\end{lstlisting}

Se presentan a continuación el mapa completo de la red y la distribución de viajes por ruta:

\begin{figure}[H]
  \centering
  \includegraphicsifexists[width=0.8\textwidth]{%
    ../static_analysis/queries/results/route_visualization/route_visualization_map.png}
  \caption{Mapa de la red de rutas de Vancouver a partir de GTFS estático.}
  \label{fig:route-map}
\end{figure}

\begin{figure}[H]
  \centering
  \includegraphicsifexists[width=0.75\textwidth]{%
    ../static_analysis/queries/results/route_visualization/route_trip_distribution.png}
  \caption{Distribución de viajes diarios por ruta de colectivo.}
  \label{fig:route-trip-stats}
\end{figure}

El mapa evidencia una red especialmente densa en el centro de Vancouver y en
los corredores troncales hacia el este y el sur. Podemos ver el color amarillo abundante en el mapa, lo cual nos indica que las únicas rutas para analizar son las de buses. Por eso, nuestros análisis estarán restringidos a las rutas de buses.
El gráfico de barras muestra que un subconjunto reducido de rutas concentra gran parte de los viajes diarios,
mientras que la mayoría opera con menor frecuencia, típicamente en zonas periféricas.

\subsection{Densidad de rutas por segmento}

Para estudiar la superposición de recorridos sobre el espacio urbano se construye 
una vista materializada de densidad de rutas por segmento, agrupando los tramos
por geometría y contando las rutas distintas que pasan por cada uno.

\begin{lstlisting}[language=SQL]
DROP MATERIALIZED VIEW IF EXISTS segment_route_density;
CREATE MATERIALIZED VIEW segment_route_density AS
SELECT
  stop1_id || stop2_id AS segment_id,
  seg_geom,
  COUNT(DISTINCT route_id) AS num_routes
FROM route_segments
WHERE seg_geom IS NOT NULL
GROUP BY stop1_id, stop2_id, seg_geom;
\end{lstlisting}

Los resultados visuales derivados de esta vista son:

\begin{figure}[H]
  \centering
  \includegraphicsifexists[width=0.75\textwidth]{%
    ../static_analysis/queries/results/route_density/route_density_histogram.png}
  \caption{Histograma de cantidad de rutas por segmento.}
  \label{fig:route-density-hist}
\end{figure}

\begin{figure}[H]
  \centering
  \includegraphicsifexists[width=0.75\textwidth]{%
    ../static_analysis/queries/results/route_density/route_density_map.png}
  \caption{Mapa de densidad de rutas sobre la red de segmentos
  (verde = 1--2 rutas; rojo = 3 o más rutas por segmento).}
  \label{fig:route-density-map}
\end{figure}

El histograma evidencia que la mayoría de segmentos tiene pocas rutas superpuestas,
mientras que un número reducido de corredores concentra un alto nivel de servicio.
En el mapa estos corredores de alta densidad aparecen en rojo alrededor del centro
de la ciudad y en ejes troncales, mientras que las zonas periféricas se observan
principalmente en verde, con menor superposición de recorridos.

\subsection{Velocidades programadas por segmento}

La velocidad promedio por segmento y por ruta se obtiene combinando la geometría de los segmentos
con los horarios de llegada previstos. Esta información se materializa en una vista que calcula
la velocidad en km/h para cada tramo entre paradas.

\begin{lstlisting}[language=SQL]
DROP MATERIALIZED VIEW IF EXISTS schedule_speeds;
CREATE MATERIALIZED VIEW schedule_speeds AS
SELECT 
  s.route_id || stop1_sequence || stop2_sequence AS id,
  AVG(seg_length / EXTRACT(EPOCH FROM 
      (stop2_arrival_time - stop1_arrival_time)) * 3.6) AS speed_kmh,
  seg_geom
FROM route_segments s
WHERE stop2_arrival_time <> stop1_arrival_time
  AND seg_length > 0
GROUP BY s.route_id, stop1_sequence, stop2_sequence, seg_geom;
\end{lstlisting}

A continuación se muestran los mapas de velocidad sobre la red y el histograma de distribución:

\begin{figure}[H]
  \centering
  \includegraphicsifexists[width=0.8\textwidth]{%
    ../static_analysis/queries/results/speed_analysis/speed_analysis_map.png}
  \caption{Mapa de velocidades programadas sobre la red de segmentos
  (3.5--19~km/h = verde punteado; 19--23~km/h = verde; 23--30~km/h = amarillo;
  $>$30~km/h = rojo).}
  \label{fig:speed-map}
\end{figure}

\begin{figure}[H]
  \centering
  \includegraphicsifexists[width=0.8\textwidth]{%
    ../static_analysis/queries/results/speed_analysis/speed_analysis_map_slow_zone.png}
  \caption{Detalle de zona más céntrica de la ciudad (más congestión).}
  \label{fig:speed-map-slow}
\end{figure}

\begin{figure}[H]
  \centering
  \includegraphicsifexists[width=0.75\textwidth]{%
    ../static_analysis/queries/results/speed_analysis/speed_distribution_histogram.png}
  \caption{Histograma de velocidades programadas por segmento.}
  \label{fig:speed-hist}
\end{figure}

Los mapas permiten localizar rápidamente los corredores con mayores velocidades
(en rojo y amarillo) frente a zonas más lentas (verde punteado), asociadas
generalmente a áreas céntricas o con alta fricción. El histograma muestra una
distribución concentrada en torno a velocidades medias, con colas hacia valores
muy bajos que corresponden a tramos fuertemente congestionados.

\subsection{Densidad poblacional y cobertura de transporte}

Para relacionar la oferta de transporte con la demanda potencial, se incorporan áreas censales
con datos de población. Se construye una vista que superpone estas áreas con la red de rutas,
calculando métricas de cobertura como la longitud de rutas por kilómetro cuadrado.

\begin{lstlisting}[language=SQL]
CREATE MATERIALIZED VIEW qgis_population_transit_overlay AS
WITH population_areas AS (
  SELECT
    id,
    geom,
    population_density,
    CAST(pop AS double precision) AS population,
    CAST(a AS double precision) AS area_km2
  FROM population_density
  WHERE geom IS NOT NULL
    AND population_density IS NOT NULL
),
bus_route_segments AS (
  SELECT DISTINCT
    rs.stop1_id || rs.stop2_id AS segment_id,
    rs.seg_geom
  FROM route_segments rs
  JOIN routes r ON rs.route_id = r.route_id
  WHERE r.route_type = '3'
    AND rs.seg_geom IS NOT NULL
)
SELECT 
  pa.id,
  pa.population_density,
  COUNT(DISTINCT brs.segment_id) AS num_segments,
  COALESCE(SUM(ST_Length(
      ST_Intersection(pa.geom, brs.seg_geom)::geography)) / 1000, 0) AS route_length_km,
  pa.area_km2,
  route_length_km / NULLIF(pa.area_km2, 0) AS route_density_km_per_km2,
  pa.geom
FROM population_areas pa
LEFT JOIN bus_route_segments brs 
  ON ST_Intersects(pa.geom, brs.seg_geom)
GROUP BY pa.id, pa.population_density, pa.geom, pa.area_km2;
\end{lstlisting}

Los siguientes mapas y gráficos ilustran la relación entre población y transporte:

\begin{figure}[H]
  \centering
  \includegraphicsifexists[width=0.8\textwidth]{%
    ../static_analysis/queries/results/population_density/pupulation_map.png}
  \caption{Mapa de densidad de población por área censal.}
  \label{fig:population-map}
\end{figure}

\begin{figure}[H]
  \centering
  \includegraphicsifexists[width=0.8\textwidth]{%
    ../static_analysis/queries/results/population_density/population_denisty_with_routes_map.png}
  \caption{Superposición de densidad de población y red de rutas de colectivo.}
  \label{fig:population-routes-map}
\end{figure}

\begin{figure}[H]
  \centering
  \includegraphicsifexists[width=0.75\textwidth]{%
    ../static_analysis/queries/results/population_density/coverage_by_density_category.png}
  \caption{Cobertura de transporte por categoría de densidad poblacional.}
  \label{fig:population-coverage}
\end{figure}

El primer mapa destaca las zonas de mayor concentración poblacional, mientras que
la superposición con las rutas permite identificar corredores bien servidos frente
a “manchas” densas con menor cobertura. En conjunto, los mapas y gráficos muestran
que las zonas céntricas concentran tanto mayor densidad poblacional como mejor
cobertura de transporte, mientras que algunas áreas densas en la periferia aparecen
con niveles de servicio más limitados.

\subsection{Accesibilidad local a estadios}

Los estadios se modelan como puntos de interés y se analiza la oferta de transporte
en un radio de 600~m. La consulta identifica las paradas y rutas cercanas, generando
una vista materializada para evaluar la accesibilidad local de cada recinto.

\begin{lstlisting}[language=SQL]
CREATE TABLE IF NOT EXISTS football_stadiums (
  id serial PRIMARY KEY,
  name text NOT NULL,
  team text,
  latitude float,
  longitude float,
  geom geometry(Point, 4326)
);

CREATE MATERIALIZED VIEW qgis_stadium_proximity AS
SELECT 
  s.name AS stadium_name,
  s.team,
  st.stop_id,
  st.stop_name,
  ST_DistanceSphere(s.geom, st.stop_loc::geometry) AS distance_m,
  st.stop_loc::geometry AS geom
FROM football_stadiums s
JOIN stops st 
  ON ST_DistanceSphere(s.geom, st.stop_loc::geometry) <= 600;
\end{lstlisting}

El análisis de proximidad produce los siguientes mapas y gráficos de oferta:

\begin{figure}[H]
  \centering
  \includegraphicsifexists[width=0.8\textwidth]{%
    ../static_analysis/queries/results/stadium_proximity/stadium_map.png}
  \caption{Mapa de estadios (naranja) y red de paradas (violeta) en un radio de 600~m.}
  \label{fig:stadium-map}
\end{figure}

\begin{figure}[H]
  \centering
  \includegraphicsifexists[width=0.75\textwidth]{%
    ../static_analysis/queries/results/stadium_proximity/stadium_stops_and_routes.png}
  \caption{Paradas y rutas que sirven a cada estadio dentro de 600~m.}
  \label{fig:stadium-stops}
\end{figure}

\begin{figure}[H]
  \centering
  \includegraphicsifexists[width=0.75\textwidth]{%
    ../static_analysis/queries/results/stadium_proximity/stadium_trips_per_day.png}
  \caption{Viajes diarios que pasan cerca de cada estadio (radio de 600~m).}
  \label{fig:stadium-trips}
\end{figure}

El mapa muestra que BC Place y Rogers Arena se encuentran en un nodo concentrado
de paradas y rutas de colectivo, mientras que otros recintos presentan menor
densidad de servicio en su entorno inmediato. El gráfico de barras confirma
que los estadios céntricos reciben significativamente más viajes diarios que
aquellos ubicados en zonas residenciales o periféricas.

Es interesante ver que aunque BC Place y Rogers Arena se encuentran muy cerca, su pequeña diferencia en distancia hace que BC Place reciba muchos menos viajes diarios.

\subsection{Conectividad entre estadios y áreas de alta densidad}

Finalmente se analiza la conectividad directa entre los estadios y las zonas de
alta densidad poblacional. La consulta mide cuántas áreas densas alcanza cada estadio
y genera geometrías completas de las rutas que realizan esta conexión.

\begin{lstlisting}[language=SQL]
WITH stadium_connectivity_stats AS (
  SELECT 
    s.id AS stadium_id,
    s.name AS stadium_name,
    s.team,
    s.geom AS stadium_geom,
    COUNT(DISTINCT stc.density_area_id) AS num_high_density_areas_connected,
    COALESCE(SUM(stc.population), 0) AS total_population_connected,
    COALESCE(SUM(rs.total_route_segments), 0) AS total_connecting_segments,
    COALESCE(SUM(rs.total_route_length_km), 0) AS total_route_length_km
  FROM football_stadiums s
  LEFT JOIN stadium_to_density_connectivity stc 
    ON s.id = stc.stadium_id
  LEFT JOIN route_statistics rs 
    ON stc.route_id = rs.route_id
  GROUP BY s.id, s.name, s.team, s.geom
)
CREATE MATERIALIZED VIEW qgis_stadium_population_overlay AS
SELECT
  stadium_name,
  team,
  num_high_density_areas_connected,
  total_population_connected,
  total_connecting_segments,
  total_route_length_km,
  stadium_geom AS geom
FROM stadium_connectivity_stats;
\end{lstlisting}

La conectividad con áreas densas se visualiza en los siguientes gráficos y mapas:

\begin{figure}[H]
  \centering
  \includegraphicsifexists[width=0.75\textwidth]{%
    ../static_analysis/queries/results/stadium_population/stadium_connected_population.png}
  \caption{Población en áreas densas conectadas a cada estadio.}
  \label{fig:stadium-connected-pop}
\end{figure}

\begin{figure}[H]
  \centering
  \includegraphicsifexists[width=0.8\textwidth]{%
    ../static_analysis/queries/results/stadium_population/stadium_high_density_population_routes.png}
  \caption{Rutas de colectivo que conectan estadios con áreas de alta densidad poblacional.}
  \label{fig:stadium-density-routes}
\end{figure}

Los gráficos y mapas muestran, por un lado, qué estadios se asocian con mayor
población en áreas de alta densidad y, por otro, los corredores principales que
materializan esas conexiones sobre la red de transporte.

\section{Análisis con GTFS-Realtime}

Los análisis sobre datos en tiempo real se basan en las trayectorias \emph{map-matched}
almacenadas en \texttt{realtime\_trips\_mdb}, las cuales se comparan con sus equivalentes
programados en \texttt{scheduled\_trips\_mdb}. Los resultados presentados en esta sección
se basan en datos recolectados el 7 de diciembre de 2025 (fin de semana, domingo) entre las 7am y las 11am (hora local de Vancouver) y
datos del 8 de diciembre de 2025 (dia de semana, lunes) entre las 8am y las 11am (hora local de Vancouver).

\subsection{Velocidad real vs.\ velocidad programada}

La comparación de velocidades se realiza mediante una vista materializada que calcula
las velocidades programadas y observadas para cada segmento, permitiendo identificar
desviaciones sistemáticas en la operación.

\begin{lstlisting}[language=SQL]
DROP MATERIALIZED VIEW IF EXISTS realtime_speed_comparison;
CREATE MATERIALIZED VIEW realtime_speed_comparison AS
WITH with_next_stop AS (
    SELECT
        d.trip_instance_id,
        d.trip_id,
        d.route_id,
        d.service_date,
        d.stop_sequence,
        d.stop_id,
        d.actual_arrival,
        LEAD(d.stop_sequence) OVER w AS next_stop_sequence,
        LEAD(d.stop_id) OVER w AS next_stop_id,
        LEAD(d.actual_arrival) OVER w AS next_actual_arrival
    FROM rt_trip_updates_deduped d
    WINDOW w AS (PARTITION BY d.trip_instance_id ORDER BY d.stop_sequence)
)
SELECT
    w.trip_instance_id,
    w.trip_id,
    r.route_short_name,
    w.route_id,
    rs.seg_length AS segment_length_m,
    EXTRACT(EPOCH FROM (rs.stop2_arrival_time - rs.stop1_arrival_time)) AS scheduled_seconds,
    EXTRACT(EPOCH FROM (w.next_actual_arrival - w.actual_arrival)) AS actual_seconds,
    (rs.seg_length / NULLIF(EXTRACT(EPOCH FROM (rs.stop2_arrival_time - rs.stop1_arrival_time)), 0) * 3.6) AS scheduled_speed_kmh,
    (rs.seg_length / NULLIF(EXTRACT(EPOCH FROM (w.next_actual_arrival - w.actual_arrival)), 0) * 3.6) AS actual_speed_kmh
FROM with_next_stop w
JOIN route_segments rs
    ON rs.trip_id = w.trip_id
    AND rs.stop1_sequence = w.stop_sequence
LEFT JOIN routes r ON r.route_id = w.route_id
WHERE w.next_actual_arrival IS NOT NULL
  AND rs.seg_length > 10
  AND EXTRACT(EPOCH FROM (rs.stop2_arrival_time - rs.stop1_arrival_time)) > 0
  AND EXTRACT(EPOCH FROM (w.next_actual_arrival - w.actual_arrival)) > 0;
\end{lstlisting}

La comparación espacial y estadística se resume en las siguientes figuras:

\begin{figure}[H]
  \centering
  \includegraphicsifexists[width=0.8\textwidth]{%
    ../realtime_analysis/queries/results/speed_vs_schedule/speed_difference_map.png}
  \caption{Mapa de diferencia de velocidad real vs.\ programada
  (rojo = más lento que lo programado, verde $\approx$ en línea con lo programado,
  amarillo = más rápido).}
  \label{fig:realtime-speed-diff-map}
\end{figure}

\begin{figure}[H]
  \centering
  \includegraphicsifexists[width=0.9\textwidth]{%
    ../realtime_analysis/queries/results/speed_vs_schedule/speed_difference.png}
  \caption{Distribución de la diferencia de velocidad (real $-$ programada).}
  \label{fig:realtime-speed-diff}
\end{figure}

En este mapa se identifican rápidamente corredores donde los buses circulan
de forma sistemáticamente más lenta que lo previsto (en rojo), así como tramos
que tienden a adelantarse sobre el horario (en amarillo). El histograma de diferencias
muestra una distribución aproximadamente unimodal, con una masa importante de segmentos
con velocidades similares a las previstas y colas que capturan tanto episodios de
fuerte sobreejecución como segmentos marcadamente más lentos que el plan.

\begin{figure}[H]
  \centering
  \includegraphicsifexists[width=0.75\textwidth]{%
    ../realtime_analysis/queries/results/speed_vs_schedule/speed_by_day_type.png}
  \caption{Comparación de velocidades programadas vs.\ observadas por tipo de día
  (día de semana vs.\ fin de semana).}
  \label{fig:realtime-speed-day-type}
\end{figure}

Con los datos obtenidos,
tenemos en cuenta tanto los fines de semana como los días de semana para poder tener una vista
completa del funcionamiento del servicio. El gráfico muestra que las velocidades promedio son
mayores durante los fines de semana, tanto programadas como observadas, lo cual es consistente
con menor congestión vehicular en esos días.

\subsection{Desempeño de horarios (schedule times)}

La comparación de horarios programados vs.\ observados se realiza mediante una vista
que registra para cada parada la hora prevista y la hora efectivamente observada,
calculando la demora en minutos para cada evento de llegada.

\begin{lstlisting}[language=SQL]
DROP MATERIALIZED VIEW IF EXISTS realtime_schedule_times;
CREATE MATERIALIZED VIEW realtime_schedule_times AS
SELECT
    d.trip_instance_id,
    d.trip_id,
    r.route_short_name,
    r.route_long_name,
    d.route_id,
    d.service_date,
    d.stop_sequence,
    d.stop_id,
    s.stop_name,
    ts.arrival_time AS scheduled_arrival_interval,
    d.actual_arrival,
    d.actual_departure,
    d.arrival_delay_seconds,
    d.departure_delay_seconds,
    d.arrival_delay_seconds / 60.0 AS delay_minutes,
    EXTRACT(hour FROM d.actual_arrival) AS hour_of_day,
    EXTRACT(dow FROM d.actual_arrival) AS day_of_week,
    CASE 
        WHEN EXTRACT(dow FROM d.actual_arrival) IN (0, 6) THEN 'Weekend'
        ELSE 'Weekday'
    END AS day_type
FROM rt_trip_updates_deduped d
JOIN routes r ON r.route_id = d.route_id
LEFT JOIN stops s ON s.stop_id = d.stop_id
LEFT JOIN transit_stops ts 
    ON ts.trip_id = d.trip_id 
    AND ts.stop_sequence = d.stop_sequence
WHERE d.arrival_delay_seconds IS NOT NULL;
\end{lstlisting}

El desempeño de puntualidad se ilustra mediante un mapa y un histograma de demoras:

\begin{figure}[H]
  \centering
  \includegraphicsifexists[width=0.8\textwidth]{%
    ../realtime_analysis/queries/results/schedule_times/delay_map.png}
  \caption{Mapa de desempeño horario en paradas
  (verde = dentro de $\pm$3~min, amarillo = $\pm$2--7~min,
  rojo = más de $\pm$7~min).}
  \label{fig:realtime-delay-map}
\end{figure}

\begin{figure}[H]
  \centering
  \includegraphicsifexists[width=0.8\textwidth]{%
    ../realtime_analysis/queries/results/delay_segments/delay_segments_map.png}
  \caption{Mapa de demoras por segmento (entre paradas consecutivas)
  (verde = dentro de $\pm$3~min, amarillo = $\pm$2--7~min,
  rojo = más de $\pm$7~min).}
  \label{fig:realtime-delay-segments-map}
\end{figure}

\begin{figure}[H]
  \centering
  \includegraphicsifexists[width=0.75\textwidth]{%
    ../realtime_analysis/queries/results/schedule_times/delay_histogram.png}
  \caption{Distribución de demoras respecto al horario programado.}
  \label{fig:realtime-delay-hist}
\end{figure}

El primer mapa muestra las demoras acumuladas en cada parada (demora total desde el inicio del viaje),
mientras que el segundo mapa muestra las demoras incrementales por segmento (demora acumulada durante
el trayecto entre dos paradas consecutivas). Ambos mapas permiten ubicar espacialmente los puntos donde el sistema opera de forma
mayormente puntual (en verde) frente a corredores donde predominan los retrasos
o adelantos marcados. Los gráficos muestran una cola hacia valores positivos que
indica un muy leve predominio de demoras por encima del horario programado, especialmente en calles troncales. Sin embargo, podemos ver 
que los horarios se suelen cumplir con precisión, observando que la mayoría del mapa se encuentra en verde.

\subsection{Regularidad de headways y \emph{bus bunching}}

El análisis de headways calcula los intervalos entre vehículos consecutivos en una
misma parada y ruta. La vista materializada permite clasificar estos intervalos en
categorías de regularidad y detectar fenómenos de \emph{bus bunching}.

\begin{lstlisting}[language=SQL]
DROP MATERIALIZED VIEW IF EXISTS realtime_headway_stats;
CREATE MATERIALIZED VIEW realtime_headway_stats AS
WITH stop_arrivals AS (
    SELECT
        rtu.route_id,
        r.route_short_name,
        rtu.stop_id,
        s.stop_name,
        rtu.trip_instance_id,
        rtu.trip_id,
        rtu.arrival_time,
        EXTRACT(hour FROM rtu.arrival_time) AS hour_of_day,
        EXTRACT(dow FROM rtu.arrival_time) AS day_of_week,
        CASE 
            WHEN EXTRACT(dow FROM rtu.arrival_time) IN (0, 6) THEN 'Weekend'
            ELSE 'Weekday'
        END AS day_type
    FROM rt_trip_updates rtu
    JOIN routes r ON r.route_id = rtu.route_id
    LEFT JOIN stops s ON s.stop_id = rtu.stop_id
    WHERE rtu.arrival_time IS NOT NULL
      AND rtu.stop_id IS NOT NULL
),
with_prev AS (
    SELECT
        *,
        LAG(arrival_time) OVER (
            PARTITION BY route_id, stop_id
            ORDER BY arrival_time
        ) AS prev_arrival,
        LAG(trip_instance_id) OVER (
            PARTITION BY route_id, stop_id
            ORDER BY arrival_time
        ) AS prev_trip_instance_id
    FROM stop_arrivals
)
SELECT
    route_id,
    route_short_name,
    stop_id,
    stop_name,
    trip_instance_id,
    prev_trip_instance_id,
    arrival_time,
    prev_arrival,
    EXTRACT(EPOCH FROM (arrival_time - prev_arrival)) / 60.0 AS headway_minutes,
    hour_of_day,
    day_of_week,
    day_type
FROM with_prev
WHERE prev_arrival IS NOT NULL
  AND trip_instance_id != prev_trip_instance_id
  AND EXTRACT(EPOCH FROM (arrival_time - prev_arrival)) > 0
  AND EXTRACT(EPOCH FROM (arrival_time - prev_arrival)) < 7200;
\end{lstlisting}

La regularidad del servicio se visualiza a través de la distribución de headways
y su clasificación por categorías:

\begin{figure}[H]
  \centering
  \includegraphicsifexists[width=0.75\textwidth]{%
    ../realtime_analysis/queries/results/headway_analysis/headway_distribution.png}
  \caption{Distribución de headways entre vehículos.}
  \label{fig:realtime-headway-dist}
\end{figure}

\begin{figure}[H]
  \centering
  \includegraphicsifexists[width=0.75\textwidth]{%
    ../realtime_analysis/queries/results/headway_analysis/headway_categories.png}
  \caption{Proporción de headways por categoría de regularidad.}
  \label{fig:realtime-headway-categories}
\end{figure}

\begin{figure}[H]
  \centering
  \includegraphicsifexists[width=0.75\textwidth]{%
    ../realtime_analysis/queries/results/headway_analysis/headway_by_day_type.png}
  \caption{Comparación de headways y tasa de \emph{bunching} por tipo de día
  (día de semana vs.\ fin de semana).}
  \label{fig:realtime-headway-day-type}
\end{figure}

Los gráficos muestran la distribución de intervalos entre vehículos y la
proporción de headways en cada categoría por ruta, lo que
permite cuantificar la incidencia del \emph{bus bunching} y de las brechas de
servicio prolongadas. La comparación por tipo de día revela que los fines de semana
presentan headways promedio más largos y menor incidencia de \emph{bunching}, reflejando
una menor frecuencia de servicio durante esos días. Debido a la hora y dias de los datos, creemos que el headway analizado es representativo
del servicio en su totalidad.

\subsection{Headway observado vs.\ headway programado}

Se creó la vista \texttt{qgis\_realtime\_headway\_vs\_schedule} para comparar el
headway observado en cada parada con el headway programado. Se filtran pares
parada--ruta con al menos tres observaciones y se calculan:
headway observado, headway programado, diferencia (observado $-$ programado) y
tasas de \emph{bunching} y \emph{gaps}. Con una muestra amplia de rutas y
paradas, se observa una diferencia promedio cercana a +5~min entre el headway
real y el programado.

\begin{figure}[H]
  \centering
  \includegraphicsifexists[width=0.8\textwidth]{%
    ../realtime_analysis/queries/results/headway_vs_schedule/headway_delta_distribution.png}
  \caption{Distribución de la diferencia de headway (observado $-$ programado).}
  \label{fig:headway-vs-sched-dist}
\end{figure}

La distribución muestra un corrimiento positivo: en la mayoría de paradas el
intervalo real es varios minutos mayor que el planificado. Los peores casos
aparecen en la ruta 210 (centro), con demoras de headway que rondan los 20~min
sobre lo programado. En el extremo
opuesto, algunas paradas de las rutas 531 y 602 presentan headways algo más
cortos que los programados, reflejando sobreoferta puntual.

\begin{figure}[H]
  \centering
  \includegraphicsifexists[width=0.85\textwidth]{%
    ../realtime_analysis/queries/results/headway_vs_schedule/worst_stops.png}
  \caption{Paradas con mayor headway observado vs.\ programado.}
  \label{fig:headway-vs-sched-worst}
\end{figure}

\begin{figure}[H]
  \centering
  \includegraphicsifexists[width=0.8\textwidth]{%
    ../realtime_analysis/queries/results/headway_vs_schedule/headway_vs_schedule_map.png}
  \caption{Mapa de diferencia de headway observado vs.\ programado por parada.}
  \label{fig:headway-vs-sched-map}
\end{figure}

\section{Discusión de resultados}

En los análisis estáticos se observa una red con alta concentración de rutas y
segmentos en el centro de Vancouver, donde también se registran mayores niveles
de densidad poblacional y de oferta de transporte por segmento. Los mapas de
velocidad programada permiten diferenciar corredores rápidos de tramos
particularmente lentos, mientras que la superposición con la densidad
poblacional muestra que algunas áreas densas en zonas periféricas presentan
menor longitud de recorridos cercanos por unidad de superficie. Al analizar estos datos,
podemos observar que las rutas de bus tienden a estar distribuidas y planeadas de una manera
muy coherente, ofreciendo mayores rutas y frecuencia en zonas con mayor densidad poblacional.

En torno a los estadios, los mapas y gráficos de proximidad indican que los
recintos céntricos disponen de mayor cantidad de paradas cercanas y de más
viajes diarios, y que existen rutas específicas que conectan estos puntos de
interés con las áreas de mayor densidad poblacional. Los estadios ubicados en
zonas residenciales o periféricas se asocian en general con menos población
conectada y con una red de segmentos de tránsito menos intensa. En este caso, podemos ver 
como el Pacific Coliseum, el cual se encuentra en una zona residencial, tiene menos paradas cercanas y menos viajes diarios que los estadios más céntricos.

Los resultados con datos en tiempo real muestran que, aunque muchos segmentos
mantienen velocidades cercanas a las programadas, existen corredores donde la
velocidad observada es sistemáticamente menor o mayor que la prevista. El horario y día elegido para el análisis es una buena muestra
representativa del servicio funcionando no solamente en su momento mas congestionado (hora pico un dia de semana), 
sino tambien como funciona el servicio la mayor parte del tiempo.
Los mapas y distribuciones de demoras por parada evidencian una cola hacia valores
positivos, con retrasos concentrados en ciertos corredores y horarios. Sin embargo, además de que estas diferencias
inevitablemente variarán dependiendo la hora del día, podemos observar que aún con muchos tramos 
no cumpliendo con la velocidad programada, el horario se suele cumplir con precisión.

Finalmente, el análisis de headways revela una mezcla de intervalos ``bunched'',
regulares y con grandes brechas, cuya distribución varía según la ruta. Estos datos presentan la información de que
la mayoría de los buses tienden a circular de forma muy regular, con alrededor de 10 minutos de media, 
pero con más del 30\% de los buses circulando con menos de 3 minutos de intervalo, lo cual es un buen indicador de que el servicio es eficiente.

En conjunto, los resultados estáticos y en tiempo real ofrecen
una imagen integrada de la estructura de la red, su relación con la densidad
poblacional y el desempeño operativo observado durante el periodo analizado. 
\end{document}
